%%%%%%%%%%%%%%%%%%%%%%%%%%%%%%%%%%%%%%%
% Deedy - One Page Two Column Resume
% LaTeX Template
% Version 1.2 (16/9/2014)
%
% Original author:
% Debarghya Das (http://debarghyadas.com)
%
% Original repository:
% https://github.com/deedydas/Deedy-Resume
%
% IMPORTANT: THIS TEMPLATE NEEDS TO BE COMPILED WITH XeLaTeX
%
% This template uses several fonts not included with Windows/Linux by
% default. If you get compilation errors saying a font is missing, find the line
% on which the font is used and either change it to a font included with your
% operating system or comment the line out to use the default font.
% 
%%%%%%%%%%%%%%%%%%%%%%%%%%%%%%%%%%%%%%
% 
% TODO:
% 1. Integrate biber/bibtex for article citation under publications.
% 2. Figure out a smoother way for the document to flow onto the next page.
% 3. Add styling information for a "Projects/Hacks" section.
% 4. Add location/address information
% 5. Merge OpenFont and MacFonts as a single sty with options.
% 
%%%%%%%%%%%%%%%%%%%%%%%%%%%%%%%%%%%%%%
%
% CHANGELOG:
% v1.1:
% 1. Fixed several compilation bugs with \renewcommand
% 2. Got Open-source fonts (Windows/Linux support)
% 3. Added Last Updated
% 4. Move Title styling into .sty
% 5. Commented .sty file.
%
%%%%%%%%%%%%%%%%%%%%%%%%%%%%%%%%%%%%%%%
%
% Known Issues:
% 1. Overflows onto second page if any column's contents are more than the
% vertical limit
% 2. Hacky space on the first bullet point on the second column.
%
%%%%%%%%%%%%%%%%%%%%%%%%%%%%%%%%%%%%%%


\documentclass[]{deedy-resume-openfont}
\usepackage{fancyhdr}
 
\pagestyle{fancy}
\fancyhf{}
 
\begin{document}

%%%%%%%%%%%%%%%%%%%%%%%%%%%%%%%%%%%%%%
%
%     TITLE NAME
%
%%%%%%%%%%%%%%%%%%%%%%%%%%%%%%%%%%%%%%
\namesection{Richard}{Hu}
{\faEnvelope \ \ \href{mailto:r.hu@berkeley.edu}{r.hu@berkeley.edu} \ \(\bullet\) \ \ \faPhone \ \ (909)  654-1001 \ \(\bullet\) \ \faGithub \ \ \href{https://github.com/rhu2001}{rhu2001} \ \(\bullet\) \ \faLinkedin \ \ \href{https://www.linkedin.com/in/rhu2001/}{rhu2001}}

%%%%%%%%%%%%%%%%%%%%%%%%%%%%%%%%%%%%%%
%
%     COLUMN ONE
%
%%%%%%%%%%%%%%%%%%%%%%%%%%%%%%%%%%%%%%

\begin{minipage}[t]{0.33\textwidth} 

%%%%%%%%%%%%%%%%%%%%%%%%%%%%%%%%%%%%%%
%     EDUCATION
%%%%%%%%%%%%%%%%%%%%%%%%%%%%%%%%%%%%%%

\section{Education} 

\subsection{UC Berkeley}
\descript{B.S. in Electrical Engineering}
\descript{and Computer Science}
\location{ May 2022}
College of Engineering \\
\location{ GPA: 3.9 / 4.0}

%%%%%%%%%%%%%%%%%%%%%%%%%%%%%%%%%%%%%%
%     COURSEWORK
%%%%%%%%%%%%%%%%%%%%%%%%%%%%%%%%%%%%%%

\section{Coursework}

\subsection{Berkeley}
CS 170: Efficient Algorithms and Intractable Problems \\
CS 61B: Data Structures (A+) \\
EE 126: Probability and Random Processes \\
CS 188: Introduction to Artificial Intelligence \\
EECS 70: Discrete Mathematics and Probability Theory \\
CS 61C: Machine Structures \\
\subsection{Other}
MATH 265: Linear Algebra \\
Machine Learning (\href{https://coursera.org/share/f28a9bb54a4cca7c445539ba73aa3d48}{certified by Coursera})

%%%%%%%%%%%%%%%%%%%%%%%%%%%%%%%%%%%%%%
%     SKILLS
%%%%%%%%%%%%%%%%%%%%%%%%%%%%%%%%%%%%%%

\section{Skills}

\subsection{Languages}
\location{Advanced:}
\textbullet{} Java \\
\textbullet{} Python \\
\textbullet{} \LaTeX\ \\ 
\location{Familiar:}
\textbullet{} C++ \\
\textbullet{} SQL

\subsection{Software}
\textbullet{} Git \\
\textbullet{} Unix-like operating systems

\subsection{Other}
\textbullet{} Unit and integration testing \\
\textbullet{} Statistics and probability \\
\textbullet{} Machine learning

%%%%%%%%%%%%%%%%%%%%%%%%%%%%%%%%%%%%%%
%
%     COLUMN TWO
%
%%%%%%%%%%%%%%%%%%%%%%%%%%%%%%%%%%%%%%

\end{minipage} 
\hfill
\begin{minipage}[t]{0.66\textwidth} 

%%%%%%%%%%%%%%%%%%%%%%%%%%%%%%%%%%%%%%
%     EXPERIENCE
%%%%%%%%%%%%%%%%%%%%%%%%%%%%%%%%%%%%%%

\section{Experience}

\runsubsection{Berkeley EECS Department} \\
\descript{Undergraduate Student Instructor - EECS 70}
\location{June 2020 – Present | Berkeley, CA}
\vspace{\topsep} % Hacky fix for awkward extra vertical space
\begin{tightemize}
\item Teaching discussion sections of 25 students twice a week to reinforce concepts introduced in lecture
\item Holding office hours and attending staff meetings with instructors and other TA’s weekly
\item  Creating official \LaTeX \ documents for weekly homework assignments
\end{tightemize}
\sectionsep

%%%%%%%%%%%%%%%%%%%%%%%%%%%%%%%%%%%%%%
%     PROJECTS
%%%%%%%%%%%%%%%%%%%%%%%%%%%%%%%%%%%%%%

\section{Projects}
\runsubsection{Chess AI}
\descript{| June 2020 - Present}
\begin{tightemize}
\item Currently developing a Chess AI in Java that plays using a multi-threaded Monte Carlo tree search with a random rollout policy
\item Implemented game logic and working on time and space optimizations to maximize the breadth and speed of Monte Carlo tree search
\item Developed comprehensive unit tests to debug move legality criteria and board display
\end{tightemize}
\sectionsep

\runsubsection{Lines of Action}
\descript{| March 2020 - April 2020}
\begin{tightemize}
\item Implemented Lines of Action board game in Java playable via command line or GUI using AWT and Swing
\item Optimized an alpha–beta pruning game tree search heuristic that won 2nd place in a class-wide tournament with over 450 entrants
\end{tightemize}
\sectionsep

\runsubsection{SILAS}
\descript{| October 2019 - December 2019}
\begin{tightemize}
\item Created a linear algebra command line utility using \texttt{argparse} and NumPy
\item Developed functionality for storing and retrieving matrices
\item Wrote efficient algorithms that compute row reductions, inversions, and multiplication and display each step
\end{tightemize}

\runsubsection{Hex Rockets}
\descript{| September 2018 - January 2019}
\begin{tightemize}
\item Collaborated with one friend to develop and maintain a cross-platform mobile game teaching hexadecimal arithmetic
\item Self-taught basic graphic design and a low-level Java mobile game development package libGDX
\item Received over 140 installs across iOS and Android with primarily 5-star reviews and won the Congressional App Challenge
\end{tightemize}

\end{minipage} 
\end{document}  \documentclass[]{article}
