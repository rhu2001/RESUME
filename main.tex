%-------------------------
% Resume in Latex
% Author : Sourabh Bajaj
% License : MIT
%------------------------

\documentclass[letterpaper,11pt]{article}

\usepackage{latexsym}
\usepackage[empty]{fullpage}
\usepackage{titlesec}
\usepackage{marvosym}
\usepackage[usenames,dvipsnames]{color}
\usepackage{verbatim}
\usepackage{enumitem}
\usepackage[hidelinks]{hyperref}
\usepackage{fancyhdr}
\usepackage[english]{babel}
\usepackage{tabularx}
\input{glyphtounicode}

\pagestyle{fancy}
\fancyhf{} % clear all header and footer fields
\fancyfoot{}
\renewcommand{\headrulewidth}{0pt}
\renewcommand{\footrulewidth}{0pt}

% Adjust margins
\addtolength{\oddsidemargin}{-0.5in}
\addtolength{\evensidemargin}{-0.5in}
\addtolength{\textwidth}{1in}
\addtolength{\topmargin}{-.5in}
\addtolength{\textheight}{1.0in}

\urlstyle{same}

\raggedbottom
\raggedright
\setlength{\tabcolsep}{0in}

% Sections formatting
\titleformat{\section}{
  \vspace{-4pt}\scshape\raggedright\large
}{}{0em}{}[\color{black}\titlerule \vspace{-5pt}]

% Ensure that generate pdf is machine readable/ATS parsable
\pdfgentounicode=1

%-------------------------
% Custom commands
\newcommand{\resumeItem}[2]{
  \item\small{
    \textbf{#1}{: #2 \vspace{-2pt}}
  }
}

\newcommand{\ritem}[1]{
  \item\small{
    {#1 \vspace{-2pt}}
  }
}

% Just in case someone needs a heading that does not need to be in a list
\newcommand{\resumeHeading}[4]{
    \begin{tabular*}{0.99\textwidth}[t]{l@{\extracolsep{\fill}}r}
      \textbf{#1} & #2 \\
      \textit{\small#3} & \textit{\small #4} \\
    \end{tabular*}\vspace{-5pt}
}

\newcommand{\resumeSubheading}[4]{
  \vspace{-1pt}\item
    \begin{tabular*}{0.97\textwidth}[t]{l@{\extracolsep{\fill}}r}
      \textbf{#1} & #2 \\
      \textit{\small#3} & \textit{\small #4} \\
    \end{tabular*}\vspace{-5pt}
}

\newcommand{\resumeProjectHeading}[2]{
  \vspace{-1pt}\item
    \begin{tabular*}{0.97\textwidth}[t]{l@{\extracolsep{\fill}}r}
      \textbf{#1} & \textit{\small #2} \\
    \end{tabular*}\vspace{-5pt}
}

\newcommand{\resumeSubSubheading}[2]{
    \begin{tabular*}{0.97\textwidth}{l@{\extracolsep{\fill}}r}
      \textit{\small#1} & \textit{\small #2} \\
    \end{tabular*}\vspace{-5pt}
}

\newcommand{\resumeSubItem}[2]{\resumeItem{#1}{#2}\vspace{-4pt}}

\renewcommand{\labelitemii}{$\circ$}

\newcommand{\resumeSubHeadingListStart}{\begin{itemize}[leftmargin=*]}
\newcommand{\resumeSubHeadingListEnd}{\end{itemize}}
\newcommand{\resumeItemListStart}{\begin{itemize}}
\newcommand{\resumeItemListEnd}{\end{itemize}\vspace{-5pt}}

%-------------------------------------------
%%%%%%  CV STARTS HERE  %%%%%%%%%%%%%%%%%%%%%%%%%%%%


\begin{document}

%----------HEADING-----------------
\begin{tabular*}{\textwidth}{l@{\extracolsep{\fill}}r}
  \textbf{{\huge Richard Hu}} & Email : \href{mailto:rizhu@berkeley.edu}{rizhu@berkeley.edu} \\
   & Mobile : +1 (909) 654-1001 \\
\end{tabular*}


%-----------EDUCATION-----------------
\section{Education}
    \resumeHeading
      {University of California, Berkeley}{Berkeley, CA}
      {Electrical Engineering and Computer Science B.S. | GPA: 3.95}{August 2019 -- December 2023}
    \resumeItemListStart
        \resumeItem{Courses}{Algorithms, Operating Systems, Data Structures, Machine Learning, Artificial Intelligence, Computer Architecture, Probability and Stochastic Processes, Convex Optimization, Linear Algebra}
        \resumeItem{Honors}{Dean’s List, Eta Kappa Nu (HKN) EECS Honor Society, Tau Beta Pi (TBP) Engineering Honor Society}
    \resumeItemListEnd
    


%-----------EXPERIENCE-----------------
\section{Experience}
  \resumeSubHeadingListStart
  
    \resumeSubheading
        {University of California, Berkeley}{Berkeley, CA}
          {Undergraduate Research Assistant (advised by Professor James Demmel)}{August 2021 -- Present}
          \resumeItemListStart
            \ritem{Conduct experiments with another undergraduate student using \textbf{randomized SVD}, \textbf{single precision SGEQP3.f}, and \textbf{QR decomposition} to compress model parameters for federated learning}
            \ritem{Report results and discuss next steps and ideas in weekly meetings with 3 PhD students and professor}
            \ritem{\textbf{Achieved 80\% test accuracy} on federated MNIST dataset with randomized SVD using 30 singular values}
          \resumeItemListEnd
    \resumeSubSubheading
     {Head Teaching Assistant (TA) - CS 70 Discrete Mathematics and Probability Theory}{June 2020 -- Present}
     \resumeItemListStart
        \ritem{Manage \textbf{over 50 members of course staff}, teach discussion sections of \textbf{40 students}, and coordinate course logistics with 4 other head TAs and 2 professors for a class of \textbf{over 850 students}}
        \ritem{Spearheaded course staff hiring by evaluating \textbf{over 300 applicants} and corresponding with EECS department hiring coordinators}
        \ritem{Rated \textbf{4.7 / 5} on average by students and won \textbf{Outstanding Graduate Student Instructor Award (2021)}, awarded to \textbf{top 10\% of TAs university-wide}}
      \resumeItemListEnd

    \resumeSubheading
      {Amazon}{Bellevue, WA}
      {Software Development Engineer Intern}{May 2021 -- August 2021}
      \resumeItemListStart
        \ritem{Developed internal debugging tool to rapidly store and retrieve transporter itineraries using \textbf{Java} and \textbf{Typescript}}
        \ritem{Consulted with \textbf{3 engineers} on On-Road Execution team to set up \textbf{AWS S3 buckets}, \textbf{AWS Glue Tables}, and \textbf{AWS Kinesis Firehose} delivery streams using \textbf{AWS CDK}}
        \ritem{Defined APIs to push itineraries through Firehose delivery stream to S3 buckets and query \textbf{AWS Athena} to retrieve itineraries by time range and transporter ID, and modified existing backend workflow to utilize new APIs}
        \ritem{Reduced time required for all itinerary-related debugging by \textbf{95\%}, from \textbf{20 minutes} down to less than \textbf{1 minute}}
      \resumeItemListEnd

  \resumeSubHeadingListEnd


%-----------PROJECTS-----------------
\section{Projects}
  \resumeSubHeadingListStart
    \resumeProjectHeading{m37 - Algorithmic cryptocurrency trading}{October 2021 -- Present}
      \resumeItemListStart
        \ritem{Forecast cryptocurrency k-line averages with ARIMA and GARCH models using \textbf{Jupyter Notebook}, \textbf{\texttt{numpy}}, \textbf{\texttt{statsmodels}}, and  \textbf{\texttt{arch}}}
        \ritem{Attained \textbf{0.04\% average error} on Bitcoin k-line mean forecasts with \textbf{over 80\%} of forecasts lying between the actual high and low of the k-line period}
      \resumeItemListEnd
  
    \resumeProjectHeading{Lines of Action}{March 2020 -- April 2020}
      \resumeItemListStart
        \ritem{Implemented 2-player Lines of Action board game in Java playable via terminal or GUI using AWT and Swing}
        \ritem{Researched game tree evaluation and implemented an AI based on \textbf{\underline{\href{http://citeseerx.ist.psu.edu/viewdoc/download?doi=10.1.1.4.3549&rep=rep1&type=pdf}{Winands et al. 2001}}}, winning \textbf{second place} in a course-wide tournament of over \textbf{400 competitors}}
      \resumeItemListEnd
    
    \resumeProjectHeading{Hex Rockets}{September 2018 -- January 2019}
      \resumeItemListStart
        \ritem{\textbf{Collaborated with one friend} using \textbf{low-level Java game development library} to develop and maintain a cross-platform mobile game teaching hexadecimal arithmetic }
        \ritem{\textbf{Won Congressional App Challenge} and received \textbf{over 100} installs across iOS and Android with \textbf{primarily 5-star} reviews}
      \resumeItemListEnd
  \resumeSubHeadingListEnd


\section{Skills}
    \textbf{Advanced}: Java, Python, C, NumPy, Jupyter Notebook, Git, Machine Learning, Statistics
    
    \textbf{Familiar}: C++, JavaScript, Typescript, SQL, Unix-like Operating Systems, AWS, TensorFlow, PyTorch

\end{document}